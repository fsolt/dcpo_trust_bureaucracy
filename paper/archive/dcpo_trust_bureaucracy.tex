% Options for packages loaded elsewhere
\PassOptionsToPackage{unicode}{hyperref}
\PassOptionsToPackage{hyphens}{url}
\PassOptionsToPackage{dvipsnames,svgnames,x11names}{xcolor}
\documentclass[
  12pt,
]{article}
\usepackage{xcolor}
\usepackage[margin=1in]{geometry}
\usepackage{amsmath,amssymb}
\setcounter{secnumdepth}{-\maxdimen} % remove section numbering
\usepackage{iftex}
\ifPDFTeX
  \usepackage[T1]{fontenc}
  \usepackage[utf8]{inputenc}
  \usepackage{textcomp} % provide euro and other symbols
\else % if luatex or xetex
  \usepackage{unicode-math} % this also loads fontspec
  \defaultfontfeatures{Scale=MatchLowercase}
  \defaultfontfeatures[\rmfamily]{Ligatures=TeX,Scale=1}
\fi
\usepackage{lmodern}
\ifPDFTeX\else
  % xetex/luatex font selection
\fi
% Use upquote if available, for straight quotes in verbatim environments
\IfFileExists{upquote.sty}{\usepackage{upquote}}{}
\IfFileExists{microtype.sty}{% use microtype if available
  \usepackage[]{microtype}
  \UseMicrotypeSet[protrusion]{basicmath} % disable protrusion for tt fonts
}{}
\usepackage{setspace}
\usepackage{longtable,booktabs,array}
\usepackage{calc} % for calculating minipage widths
% Correct order of tables after \paragraph or \subparagraph
\usepackage{etoolbox}
\makeatletter
\patchcmd\longtable{\par}{\if@noskipsec\mbox{}\fi\par}{}{}
\makeatother
% Allow footnotes in longtable head/foot
\IfFileExists{footnotehyper.sty}{\usepackage{footnotehyper}}{\usepackage{footnote}}
\makesavenoteenv{longtable}
\usepackage{graphicx}
\makeatletter
\newsavebox\pandoc@box
\newcommand*\pandocbounded[1]{% scales image to fit in text height/width
  \sbox\pandoc@box{#1}%
  \Gscale@div\@tempa{\textheight}{\dimexpr\ht\pandoc@box+\dp\pandoc@box\relax}%
  \Gscale@div\@tempb{\linewidth}{\wd\pandoc@box}%
  \ifdim\@tempb\p@<\@tempa\p@\let\@tempa\@tempb\fi% select the smaller of both
  \ifdim\@tempa\p@<\p@\scalebox{\@tempa}{\usebox\pandoc@box}%
  \else\usebox{\pandoc@box}%
  \fi%
}
% Set default figure placement to htbp
\def\fps@figure{htbp}
\makeatother
% definitions for citeproc citations
\NewDocumentCommand\citeproctext{}{}
\NewDocumentCommand\citeproc{mm}{%
  \begingroup\def\citeproctext{#2}\cite{#1}\endgroup}
\makeatletter
 % allow citations to break across lines
 \let\@cite@ofmt\@firstofone
 % avoid brackets around text for \cite:
 \def\@biblabel#1{}
 \def\@cite#1#2{{#1\if@tempswa , #2\fi}}
\makeatother
\newlength{\cslhangindent}
\setlength{\cslhangindent}{1.5em}
\newlength{\csllabelwidth}
\setlength{\csllabelwidth}{3em}
\newenvironment{CSLReferences}[2] % #1 hanging-indent, #2 entry-spacing
 {\begin{list}{}{%
  \setlength{\itemindent}{0pt}
  \setlength{\leftmargin}{0pt}
  \setlength{\parsep}{0pt}
  % turn on hanging indent if param 1 is 1
  \ifodd #1
   \setlength{\leftmargin}{\cslhangindent}
   \setlength{\itemindent}{-1\cslhangindent}
  \fi
  % set entry spacing
  \setlength{\itemsep}{#2\baselineskip}}}
 {\end{list}}
\usepackage{calc}
\newcommand{\CSLBlock}[1]{\hfill\break\parbox[t]{\linewidth}{\strut\ignorespaces#1\strut}}
\newcommand{\CSLLeftMargin}[1]{\parbox[t]{\csllabelwidth}{\strut#1\strut}}
\newcommand{\CSLRightInline}[1]{\parbox[t]{\linewidth - \csllabelwidth}{\strut#1\strut}}
\newcommand{\CSLIndent}[1]{\hspace{\cslhangindent}#1}
\setlength{\emergencystretch}{3em} % prevent overfull lines
\providecommand{\tightlist}{%
  \setlength{\itemsep}{0pt}\setlength{\parskip}{0pt}}
\usepackage{array}
\usepackage{caption}
\usepackage{graphicx}
\usepackage{siunitx}
\usepackage{colortbl}
\usepackage{multirow}
\usepackage{hhline}
\usepackage{calc}
\usepackage{tabularx}
\usepackage{threeparttable}
\usepackage{wrapfig}
\usepackage{fullpage}
\usepackage{lscape}
\newcommand{\blandscape}{\begin{landscape}}
\newcommand{\elandscape}{\end{landscape}}
\usepackage{titlesec}
\titleformat*{\section}{\normalsize\bfseries}
\titleformat*{\subsection}{\normalsize\itshape}
\usepackage{titling}
\usepackage{booktabs}
\usepackage{longtable}
\usepackage{array}
\usepackage{multirow}
\usepackage{wrapfig}
\usepackage{float}
\usepackage{colortbl}
\usepackage{pdflscape}
\usepackage{tabu}
\usepackage{threeparttable}
\usepackage{threeparttablex}
\usepackage[normalem]{ulem}
\usepackage{makecell}
\usepackage{xcolor}
\usepackage{bookmark}
\IfFileExists{xurl.sty}{\usepackage{xurl}}{} % add URL line breaks if available
\urlstyle{same}
\hypersetup{
  colorlinks=true,
  linkcolor={black},
  filecolor={Maroon},
  citecolor={black},
  urlcolor={Blue},
  pdfcreator={LaTeX via pandoc}}

\title{Trust in Civil Servants:\\
A Cross-National Dataset for Public Policy Research,\\
1986--2022}
\author{}
\date{\vspace{-2.5em}}

\begin{document}
\maketitle

\setstretch{1.5}
\pagenumbering{gobble}

\renewcommand{\baselinestretch}{1}
\selectfont
\maketitle
\renewcommand{\baselinestretch}{1.5}
\selectfont

\begin{abstract}
Trust in civil servants is essential for effective governance, enabling policy implementation, public service delivery, and compliance. However, the lack of comparable cross-national data on trust in bureaucracy has limited our ability to systematically examine these relationships. To address this gap, we develop the Trust in Civil Servants (TCS) dataset using an advanced latent-variable modeling technique, using 132 national and cross-national surveys from 98 countries (1986-2022). Our measures reveal variations in trust both within and between countries. We find that economic performance and public security enhance trust in the short term, whereas government quality and effectiveness have more enduring, long-term impacts on trust in civil service. The TCS dataset opens new avenues for examining connections between trust, governance quality, and complex policy challenges across different contexts.
\end{abstract}

Keywords: political trust, latent variable model, public administration, governance

\section*{Authors}\label{authors}
\addcontentsline{toc}{section}{Authors}

\begin{itemize}
\tightlist
\item
  Yuehong Cassandra Tai, ORCID:
  \url{https://orcid.org/0000-0001-7303-7443}, Assistant Research
  Professor, Center for Social Data Analytics, Pennsylvania State
  University,
  \href{mailto:yhcasstai@psu.edu}{\nolinkurl{yhcasstai@psu.edu}}
\item
  Frederick Solt, ORCID: \url{https://orcid.org/0000-0002-3154-6132},
  Professor, Department of Political Science, University of Iowa,
  \href{mailto:frederick-solt@uiowa.edu}{\nolinkurl{frederick-solt@uiowa.edu}}
\end{itemize}

\section*{Declaration of conflicting
interest}\label{declaration-of-conflicting-interest}
\addcontentsline{toc}{section}{Declaration of conflicting interest}

No potential conflict of interest was reported by the author(s).

\section*{Data availability}\label{data-availability}
\addcontentsline{toc}{section}{Data availability}

The code used to generate the dataset and conduct validation test are
openly available at:
\url{https://github.com/fsolt/dcpo_trust_bureaucracy}.

\pagebreak

\pagenumbering{arabic}

Political trust is a foundation of regime support, democratic
legitimacy, and governance (Easton, 1975; Norris et al., 2002; Wuttke et
al., 2020). It is also associated with civic duty and policy compliance
(Bargain and Aminjonov, 2020; Valgarsson et al., 2021), as well as
individuals' well-being during crises (Devine et al., 2021; Zaki et al.,
2022). For these reasons, political trust's dynamics, determinants, and
consequences have been a leading issue in political science and public
administration.

Yet despite its importance, empirical findings about whether and how
political trust matters remain mixed. While examining redistribution
policies, Devine (2024) finds negligible effects of political trust on
policy preferences in the UK and Switzerland, contrasting with Macdonald
(2021)'s evidence of trust shaping redistribution support in the United
States. When the focus shifts to trust in administrative and
implementing institutions, the evidence is also mixed: Harring (2018)
found no effect of trust in administrative institutions on environmental
policy support, while Bergquist et al. (2022) demonstrated that public
trust in implementing institutions has a stronger effect on supporting
climate change policies than other institutional trust.

These mixed findings illustrate that our empirical leverage remains
limited. Comparative research on trust is hindered by the lack of
comparable cross-national time-series measures (Tai, 2025). Existing
surveys often use different trust items and provide inconsistent
country--year coverage, which makes results difficult to compare across
studies. As a result, it is hard to evaluate competing claims across
governance contexts and over time, or to distinguish whether conflicting
results reflect real contextual heterogeneity or differences in
measurement and coverage.

This limitation is particularly consequential for research on trust in
civil servants. For public sectors, public trust is not only a
fundamental element of governance (Blind, 2007), but also a central
normative objective of public administration (Goodsell, 2006). For
policy studies, trust in civil servants deserves particular attention
because these officials directly interact with citizens, deliver public
services, and translate policies into practice (Morelock, 2021). With
adequate trust, the public more readily accepts services and complies
with policy directives (Kim, 2005: 611). Conversely, low trust can
impede officials' ability to implement policies effectively and secure
public cooperation (Van Ryzin, 2011; Yates, 1982).

Despite this importance, existing research on trust in civil servants
remains geographically and temporally constrained (Choi, 2018; Houston
et al., 2016; Morelock, 2021; Van de Walle and Migchelbrink, 2022),
limiting our ability to understand how trust dynamics shape the policy
process across different governance contexts. In addition, limited
cross-nationally comparable data makes it challenging to test competing
theories about the factors influencing trust in bureaucracy, whether it
is government performance, governance quality, or other factors
(Bouckaert, 2012; Kettl, 2000; Morelock, 2021; Van de Walle and
Migchelbrink, 2022).

To address this gap, we introduce the Trust in Civil Servants (TCS)
dataset, which leverages 132 national and cross-national surveys
covering 98 countries over 36 years (1986--2022) and applies recent
advances in latent-variable modeling of public opinion (Solt, 2020c). We
validate the TCS data by demonstrating strong correlations with
individual survey items and related measures, such as perceived
corruption and trust in other political institutions.

Using the TCS data, we conduct a cross-national time-series analysis to
examine competing theories on trust in civil servants, focusing on
government outcomes versus government quality. We find that government
outcomes, such as economic performance and public security, have
short-term effects on trust, while government quality including
effectiveness, exerts more significant, enduring effects. This
underscores that while both factors are important, the quality of
governance plays a long-term role in fostering trust in civil servants.

Our study contributes to comparative public administration and public
policy by providing valid, comparable longitudinal data on trust in
civil servants. Recent research underscores the need to explore how
trust in public institutions influence complex policy challenges like
climate change and environmental mitigation support, CO2 emissions, and
decarbonization in comparative context (Cole et al., 2024; Davidovic,
Forthcoming). The TCS dataset addresses these calls by enabling
cross-national research on how trust interacts with governance quality
and public sector performance, linking administrative practices to
governance challenges and policy outcomes. Moreover, the TCS dataset
contributes to ongoing policy debates on the role of institutional trust
in shaping policy preferences. By facilitating comparative analyses, the
TCS dataset allows researchers to explore whether and how trust
influences support for complex policies across diverse contexts.

\section{Debates on the Causes of Trust in Civil
Servants}\label{debates-on-the-causes-of-trust-in-civil-servants}

A longstanding puzzle in public administration is understanding what
explains trust in bureaucracy. One dominant theme is the belief that
higher levels of government performance lead to greater trust in civil
servants, based on the assumption that better performance correlates
with higher trust and that lower trust toward bureaucrats reflects
dissatisfaction with government performance (Yang and Holzer, 2006). A
common approach to measuring performance is through macroeconomic
outcomes, such as economic growth, unemployment rate, economic
inequality, and inflation. However, the results from studies on
macroeconomic outcomes are mixed. For example, Choi (2018) found that
GDP per capita positively affects trust in bureaucracies, while Houston
et al. (2016) did not find significant effects of GDP per capita and
inflation rate on trust in civil servants. Instead, Houston et al.
(2016) found that the unemployment rate negatively influences trust in
civil servants. Contrary to previous studies that found some evidence
for the role of government outcomes, Morelock (2021) found that none of
the outcome indicators, including GDP per capita, inflation rate,
unemployment, and the Gini index, had a significant effect on trust in
civil servants.

Amidst these mixed results regarding macroeconomic outcomes, a growing
body of literature emphasizes the role of government quality---or
process---in explaining trust in bureaucracy. Van Ryzin (2011) found
that the quality of government, measured by the World Bank's Worldwide
Governance Indicators, plays a more crucial role than government
outcomes measured by the UN's Human Development Index, which had a
negative effect in his model. Morelock (2021) also highlights the
positive role of government effectiveness, although Houston et al.
(2016) finds inconsistent role of government effectiveness. A relatively
consistent finding across studies is the significant role of corruption.
Van de Walle and Migchelbrink (2022) concluded that the perceived
absence of corruption is more impactful on trust in bureaucracy than
performance evaluations. The critical influence of perceived corruption
and corruption control on public trust in civil servants is also
supported by Houston et al. (2016) and Morelock (2021). Beyond these
findings, recent research has explored dimensions of government
performance, including transparency, agency reputation, and the
integration of input, process, and output measures. Studies show that
both public and private elite actors' trust in agencies is strongly
influenced by performance (Kappler et al., 2024). Moreover, transparency
and perceived organizational reliability have been identified as key
factors in shaping public trust (Schmidthuber et al., 2023). Despite
these advancements, variations in measures and modeling
strategies---such as whether both outcomes and quality indicators are
included in the same model---leave uncertainty about the consistency of
these results. A more standardized approach is needed to clarify these
relationships.

These mixed results also reflect limitations in comparative data,
including limited country coverage, reliance on cross-sectional rather
than dynamic analysis, and the absence of comparable measures across
countries or regions (Choi, 2018; Houston et al., 2016; Morelock, 2021;
Van de Walle and Migchelbrink, 2022; Van Ryzin, 2011). These
shortcomings hinder a deeper understanding of the relationship between
government outcomes, quality, and trust in bureaucracy.

To address these challenges, we developed the Trust in Civil Servants
(TCS) dataset, a dynamic, cross-national measure that enables rigorous
testing of competing theories about the sources of trust in civil
servants.

\section{Examining the Source Data on Trust in Civil
Servants}\label{examining-the-source-data-on-trust-in-civil-servants}

Over the past half-century, many national and cross-national surveys
have asked questions on trust attitudes toward public administrations.
Our goal is not to draw a sample of surveys, but to compile as much
feasible set of relevant survey items as possible and then synthesize
them into a comparable country--year series.

We first defined the target construct as public trust in civil servants
or public administration based on previous trust in civil servants
studies. We then undertook an extensive effort to collect and compile
relevant survey questions. This involved a systematic review of of
survey documentation and raw data files to identify items that reference
civil servants, government administrators, or public administration, and
we recorded the corresponding question identifiers, response scales,
fieldwork dates, and survey weights when available. Candidate items were
then cross-checked by two authors to resolve construct mismatches. This
process finally included 132 unique survey projects including both
cross-national globe surveys and single country surveys and spanning 125
countries over 49 year to maximize broad geographic and temporal
coverage and 27 unique candidate questions. \footnote{ The complete list
  of trust in civil servants/public administration survey items and
  included survey projects is included in online Appendix
  \nobreakspace{}(\textbf{sec-app-survey?}).}

We then processed raw files using DCPOtools (Solt et al., 2019). To
minimize the noise from the sparse data and increase comparability,
drawn from the raw data, we followed a common approach (Woo et al.,
2023) and excluded 17 survey items that were asked in fewer than five
country-years in countries surveyed at least twice. DCPOtools
standardizes formats, applies consistent recoding and weighting, and
outputs analysis data for model estimation.

Together, the survey items in the source data were asked in 98 different
countries in at least two time points over 36 years, from 1986 to 2022,
yielding a total of 1,814 country-year-item observations. If all of
these countries were surveyed in all of these years, we would have 3,528
observations per year and a total of 59,976 country-year-item
observations. However, the actual dataset is far more limited, with only
1,344 country-years containing at least some data on trust in civil
servants. This accounts for 54\% of the 2,475 country-years spanned by
our dataset. Moreover, the many different survey items employed render
these data incomparable and difficult to use together.

\begin{figure}
\centering
\pandocbounded{\includegraphics[keepaspectratio]{dcpo_trust_bureaucracy_files/figure-latex/itemcountry-1.pdf}}
\caption{Countries and Years with the Most Observations in the Source
Data \label{item_country_plots}}
\end{figure}

Consider the most frequently asked item in the data we collected, which
asks respondents whether they strongly agree, agree, disagree, or
strongly disagree with the statement ``I am going to name a number of
institutions. For each one, could you tell me how much trust you have in
them. Is it a great deal of trust, some trust, not very much trust or
none at all? Civil service.''\footnote{Question text may vary slightly
  across survey datasets, but not, roughly speaking, by more than the
  translation differences across languages found within the typical
  cross-national survey dataset. In this case, some questions ask about
  ``the public administration'' or ``government officials'' rather than
  ``the civil service,'' and some refer to ``confidence'' rather than
  ``trust.'' These words are often translated identically.} Employed by
the Arab Barometer, the Asia Europe Survey, the Asian Barometer, the
British Social Attitudes Survey, the Latino Barometer, the East Asian
Social Survey, the European Values Survey, the Italian National Election
Study, the South Asian Barometer, and the World Values Survey, this
question was asked in a total of 614 different country-years. However,
this represents only 25\% of the country-years spanned by our data,
despite being the \emph{most common} survey item. This again underscores
the sparse and often incomparable nature of the available public opinion
data on this topic.

The distribution of country-year-item observations further highlights
the limitations of the raw dataset. As depicted in the upper left panel
of Figure \nobreakspace{}\ref{item_country_plots}, Germany, with 50
country-year-item observations, is the most represented country,
followed by Spain, Finland, the United Kingdom, and Sweden. The upper
right panel expands on this by listing the twelve countries with the
highest number of years observed, revealing overlaps and differences
from the previous group; Ireland, Italy, Austria, and Bulgaria join the
list, replacing Sweden, Slovenia, Netherlands, and France. The bottom
panel counts the countries observed in each year and reveals just how
few relevant survey items were asked before 1996. Country coverage
reached its peak in 2001, when respondents in 70 countries were asked
items about trust in civil servants.

In the next section, we describe how we leveraged this sparse and
incomparable survey data to generate complete, comparable time-series
TCS scores using a latent variable model.

\section{Estimating Trust in Civil
Servants}\label{estimating-trust-in-civil-servants}

To estimate trust in civil servants across countries and over time, we
employ the recent and suitable method for handling data that is both
incomparable and sparse: the Dynamic Comparative Public Opinion (DCPO)
model developed by Solt (2020c).\footnote{ The DCPO model provides a
  better fit to survey data than the models proposed in Claassen (2019)
  or Caughey et al. (2019; Solt, 2020c). The model put forward in McGann
  et al. (2019) depends on dense survey data unlike the sparse data on
  trust in civil servants just described. Building on all of these four
  works, Kolczynska et al. (2020) is the very most recent effort, but
  the multilevel regression and post-stratification (MRP) approach it
  offers depends both on dense survey data and on additional data
  describing population characteristics, so it too is inappropriate for
  our purposes here.} DCPO treats trust in civil servants as a latent
trait that is not directly observable but can be inferred from
respondents' answer to multiple relevant survey questions, accounting
for differences in question wording and response scales. DCPO has been
applied in recent peer-reviewed work to produce comparable
public-opinion time series from sparse and incomparable cross-national
surveys (Hu and Solt, 2025; e.g., Tai et al., 2024; Woo et al., 2023).
Formally, DCPO is a population-level two-parameter ordinal logistic item
response theory (IRT) model with country-specific item-bias terms,
addressing the two principal challenges posed by our source data:
incomparability and sparsity.

DCPO models the total number of survey responses expressing at least as
much trust in civil servants as response category \(r\) to each question
\(q\) in country \(k\) at time \(t\), \(y_{ktqr}\), out of the total
number of respondents surveyed, \(n_{ktqr}\), using the beta-binomial
distribution to accommodate overdispersion beyond sampling error:

\begin{equation}
a_{ktqr} = \phi\eta_{ktqr} \label{eq:bb_a}
\end{equation} \begin{equation}
b_{ktqr} = \phi(1 - \eta_{ktqr}) \label{eq:bb_b}
\end{equation} \begin{equation}
y_{ktqr} \sim \textrm{BetaBinomial}(n_{ktqr}, a_{ktqr}, b_{ktqr}) \label{eq:betabinomial}
\end{equation}

where \(\eta_{ktqr}\) is the expected probability that a random person
in country \(k\) at time \(t\) answers question \(q\) in a category at
least as positive as response \(r\) and \(\phi\) represents an overall
dispersion parameter.

This expected probability, \(\eta_{ktqr}\), is linked to latent trust
via a population-level ordinal IRT measurement equation:

\begin{equation}
\eta_{ktqr} = \textrm{logit}^{-1}(\frac{\bar{\theta'}_{kt} - (\beta_{qr} + \delta_{kq})}{\sqrt{\alpha_{q}^2 + (1.7*\sigma_{kt})^2}}) \label{eq:dcpo}
\end{equation}

Here \(\bar{\theta'}_{kt}\) is the (unbounded) country--year mean of
latent trust in civil servants and \(\sigma_{kt}\) is country--year
standard deviation. The \(\beta_{qr}\) parameter captures the difficulty
of endorsing category \(r\) to question \(q\), \(\alpha_{q}\) captures
the dispersion/noisiness of question \(q\) in measuring trust in civil
servants, and \(\delta_{kq}\) term represents country-specific item bias
that allows the same survey item to function differently across
countries.

To address sparsity over time, DCPO places random-walk priors on the
latent country--year parameters and estimates are therefore smoother in
data-rich periods and more uncertain in data-sparse periods. Together,
DCPO estimates latent trust and smooths estimates simultaneously in a
single Bayesian model, with a random-walk evolution prior providing
parsimonious regularization across sparse time series and naturally
expanding uncertainty where coverage is thin. For additional details,
see Appendix \nobreakspace{}(\textbf{sec-app-dcpo?}) and Solt (2020c:
3--8).

Intuitively, DCPO addresses item incomparability through the
\emph{difficulty} and \emph{dispersion} parameters. \emph{Difficulty}
captures how much trust in civil servants is indicated by a given
response. For example, strongly agreeing with the statement ``Most
government administrators (civil servants) can be trusted to do what is
best for the country,'' exhibits more trust in civil servants than
simply agreeing, which shows more trust than responding ``disagree,''
which in turn is a more trusting response than ``strongly disagree.''
The same logic also applies across questions. Expressing ``great trust''
in civil servants ``to look after your interests'' likely expresses even
more trust than just strongly agreeing that civil servants can be
trusted to do what is right. \emph{Dispersion} captures how tightly a
question's responses map onto the latent trait: questions with lower
dispersion provide a stronger signal of changes in trust, while noisier
questions contribute weaker information. Together, by estimating
difficulty and dispersion, DCPO maps responses from different questions
and surveys onto a single latent scale, yielding comparable
country--year estimates.

Finally, the random-walk priors smooth trust estimates over time within
each country. A given year's trust level is modeled as the previous
year's estimate plus a random shock. This approach allows the generation
of estimates even for years with little or no data, with uncertainty
increasing as the time gap between observed years grows. We incorporate
this uncertainty in downstream analyses as described below.

\begin{figure}
\centering
\pandocbounded{\includegraphics[keepaspectratio]{dcpo_trust_bureaucracy_files/figure-latex/cs-1.pdf}}
\caption{TCS Scores, Most Recent Available Year \label{cs_mry}}
\end{figure}

We estimated the model using the \texttt{DCPOtools} package for R (Solt,
2020a), running four chains for 1,000 iterations each and discarding the
first half as warmup, which left us with 2,000 samples. The \(\hat{R}\)
diagnostic had a maximum value of 1.01, indicating that the model
converged. The dispersion parameters of the survey items indicate that
all of our source data items load well on the latent variable (see
Appendix \nobreakspace{}(\textbf{sec-app-survey?})).

The result is estimates, in all 2,475 country-years spanned by the
source data, of public trust in civil servants, what we call TCS scores.
Figure\nobreakspace{}\ref{cs_mry} displays the most recent available TCS
score for each of the 98 countries and territories in the dataset.

Several Asian countries, such as China and Singapore, dominate the top
of the list. Several Southeast Asian countries also rank highly. This
pattern aligns with OECD/ADB reports documenting high self-reported
satisfaction with core public services in the region in recent years
(OECD and Asian Development Bank, 2019, 2025). Several democracies, such
as Switzerland, Norway, Denmark, and Finland, also rank highly. On the
other hand, the latest scores for Peru, Mexico, Honduras, Guatemala, and
Ecuador have them as the places where the public has the lowest trust
toward civil servants.

For transparency, we release the full set of country--year estimates and
associated uncertainty. We treat unexpected estimates as prompts for
further inquiry rather than definitive substantive claims, and we
suggest that scholars interpret cross-national differences cautiously,
especially for countries with sparse item coverage, where uncertainty is
larger.

\begin{figure}
\centering
\pandocbounded{\includegraphics[keepaspectratio]{dcpo_trust_bureaucracy_files/figure-latex/ts-1.pdf}}
\caption{TCS Scores Over Time Within Selected Countries
\label{ts_plots}}
\end{figure}

We then show the changes of TCS over time in sixteen countries in
Figure\nobreakspace{}\ref{ts_plots}. As displayed in
Figure\nobreakspace{}\ref{cs_mry}, the dataset covers a wide geographic
breadth, allowing comparative studies of countries and regions too often
neglected (see Wilson and Knutsen, 2022).
Figure\nobreakspace{}\ref{ts_plots} also shows that trust in civil
servants has risen prominently in some countries, such as Germany and
New Zealand, while remaining fairly constant over time in others, like
Greece and Australia. In contrast, TCS scores have fallen steadily in
countries such as South Korea and the United States. Some countries
exhibit fluctuations, as seen in the United Kingdom, where trust has
advanced and retreated, or the Philippines, where trust has declined and
later recovered. Together, the differences within countries over time
and the differences across countries present a challenge to theories on
the causes and consequences of trust in civil servants.

\section{Validating Trust in Civil
Servants}\label{validating-trust-in-civil-servants}

\begin{figure}
\centering
\pandocbounded{\includegraphics[keepaspectratio]{dcpo_trust_bureaucracy_files/figure-latex/internalval-1.pdf}}
\caption{Convergent Validation: Correlations Between TSC Scores and
Individual TSC Source-Data Survey Items \label{internal_val}}
\end{figure}

Before using these estimates in analysis, we validate our trust civil
service score through convergent validation and construct validation,
since validation tests of cross-national latent variables are crucially
important (see, e.g., Hu et al., 2025).
Figure\nobreakspace{}\ref{internal_val} shows the measure's validity in
tests of convergent validation that tests whether a measure is
empirically associated with alternative indicators of the same concept
(Adcock and Collier, 2001: 540). We started with `internal' convergent
validation test (see, e.g., Caughey et al., 2019: 689; Solt, 2020c: 10)
by comparing our TCS score with individual items from source-data to
generate them.

The left panel in Figure\nobreakspace{}\ref{internal_val} shows a
scatterplot of country-years in which the TCS scores are plotted against
the percentage of respondents who expressed ``a quite a lot''or ``a
great deal'' of trust in response to the question: ``I am going to name
a number of institutions. For each one, could you tell me how much trust
you have in them. Is it a great deal of trust, some trust, not very much
trust or none at all? Civil service.'' The strong correlation (R = 0.93)
indicates that TCS scores effectively capture variations in trust in
civil service across country-years.

The middle panel plots our TCS score against the percentage who
responded ``Tend to trust.'' to the question, ``I would like to ask you
a question about how much trust you have in certain institutions. For
each of the following institutions, please tell me if you tend to trust
it or tend not to trust it: Public administration in (OUR COUNTRY)'' in
the Eurobarometer 96.3 January-February 2022 module. This question is
asked in the most countries, and the strong correlation demonstrates the
broad applicability of the TCS scores in capturing trust across diverse
contexts.

Finally, the right panel compares the trend of the longest item that has
been asked since 1973 in U.S.General Social Survey, ``I am going to name
some institutions in this country. As far as the people running these
institutions are concerned, would you say you have a great deal of
confidence, only some confidence, or hardly any confidence at all in
them? Executive branch of the federal government.'' to the trend of the
TCS scores. The TCS scores align with trends in trust in the executive
branch over time, effectively capturing historical changes.

\begin{figure}
\centering
\pandocbounded{\includegraphics[keepaspectratio]{dcpo_trust_bureaucracy_files/figure-latex/extval1-1.pdf}}
\caption{Construct Validation: Correlations Between TCS Scores and Trust
in Institutions Survey Items \label{ext_val1}}
\end{figure}

Figure\nobreakspace{}\ref{ext_val1} present three `external' convergent
validation tests, comparing TCS scores to responses to survey items that
were \emph{not} included in the source data: items that asked
respondents' confidence and trust in national government, parliament,
and judiciary in their countries. In the left panel, we plot TCS score
against data from seven rounds of World Value Survey, which asked
respondents how much they trust their national government. The center
plot shows data from European Values Surveys asking respondents'
confidence in parliament. The right presents the percentage of
respondents who expressed at least some trust in judiciary in their
country in Latinobarometro. Our measure positively correlated with all
of them, with a stronger correlation with trust in national government
and mild correlation with trust in parliament and judiciary.

There is a longstanding debate about the dimentionality of political
trust (Easton, 1965; Marien and Hooghe, 2011; Norris, 2011; Rothstein
and Stolle, 2008; Tai, 2022). Trust in civil servants has been
theoretically grouped within the same dimension as all three types of
institutional trust (Hooghe, 2011; Marien and Hooghe, 2011), or one of
them (Norris, 2011; Rothstein and Stolle, 2008; Tai, 2022). However, the
variation in correlations between TCS scores and trust in institutions
requires empirical analysis of trust's dimensions.

\begin{figure}
\centering
\pandocbounded{\includegraphics[keepaspectratio]{dcpo_trust_bureaucracy_files/figure-latex/extval2-1.pdf}}
\caption{Construct Validation: Correlations Between TCS Scores and
Corruption of Public Servants Survey Items \label{ext_val2}}
\end{figure}

We next conduct tests of construct validation in
Figure\nobreakspace{}\ref{ext_val2}. Construct validation assesses
whether a given indicator is empirically correlated with other
indicators in a way that conforms to theoretical expectations (Adcock
and Collier, 2001: 542). Corruption is often often argued as a likely
contributor to distrust in civil servants and public administration
(see, e.g., Anderson and Tverdova, 2003; Van de Walle and Migchelbrink,
2022; Van Ryzin, 2011).

The left panel compares perceived widespread of corruption, measured as
the percentage of those saying most or state authorities are involved in
corruption in seven waves of the WVS, with the TCS scores. As
anticipated, there is a clear negative relationship between the spread
of perceived corruption and the TCS scores: when there is widespread
perception of corruption in authorities, the public tends to distrust
civil servants. The similar negative correlations between TCS scores and
perceived corruption among government officials are also perceived in
the center and right panel of Figure\nobreakspace{}\ref{ext_val2}, which
used data from different regions. The center panel shows the data in
developing or newly democratic countries surveyed in the Asian
Barometer, the New Europe Barometer, and the Latinobarometro, and the
right panel displays the data in countries surveyed in the International
Social Survey Programme Citizenship module (2004, 2014).

To sum up, the evidence of construct validation of TCS scores against
the perceived extent of corruption in
Figure\nobreakspace{}\ref{ext_val2}, together with the evidence of
external validation in Figure\nobreakspace{}\ref{ext_val1} and
convergent validation in Figure \nobreakspace{}\ref{internal_val},
demonstrates the validity of the TCS scores as measures of the public's
trust in civil servants.

\section{Explaining Trust in Civil
Servants}\label{explaining-trust-in-civil-servants}

With our time-series cross-national data on trust in civil servants, we
are able to reexamine and perform a conceptual replication and
generalization (see Walker et al., 2017: 1225--1226) of the literature
debating the causes of trust in bureaucracy that we describe above. That
is, we investigate the same hypotheses as previous work using different
measurement and analysis. With many more country-years observed, we are
able to combine a wide range of contextual indicators of both outcome
and quality in a single analysis to examine the factors influencing
trust; previous studies, looking at smaller samples, are constrained to
look at these indicators just one or a few at a time. For outcome
indicators, we employ GDP per capita, inflation, unemployment, and
income inequality from 1984 to 2022 as measures of macroeconomic
performance Morelock (2021) as well as public safety with regard to
physical insecurity (see Uddin, 2025). GDP per capita and inflation data
were sourced from the International Monetary Fund, while unemployment
data were collected from the World Bank, which uses modeled
International Labour Organization estimates. Regarding income
inequality, we relied on the Standardized World Income Inequality
Database presented in Solt (2020b), specifically the Gini index of
inequality in disposable income. To measure physical insecurity, we
collected the number of intentional homicides at the country-year level
from the United Nations Office on Drugs and Crime.

For the process-oriented quality of government, we include corruption
perceptions, government effectiveness, and democracy (Choi, 2018; see,
e.g., Houston et al., 2016; Morelock, 2021). To capture the perceived
level of corruption, we use the Corruption Perceptions Index from
Transparency International, covering the years 1995 to 2022. The World
Bank's Worldwide Governance Indicators provides its measure of
Government Effectiveness, which reflects the overall quality of public
services, the civil service, and policy formulation and implementation.
And to account for the effect of democratic development on trust, we
included the Liberal Democracy Index from the V-Dem dataset (Coppedge et
al., 2023; Pemstein et al., 2023).

We adopted a Bayesian multilevel model with varying intercepts for each
country and each year. The varying intercepts for each country account
for heteroskedasticity across countries, while those for each year
account for `time shocks' that impact all countries simultaneously (Shor
et al., 2007). To differentiate between short-term and historical
effects, we used the `within-between random effects' specification as
described by Bell and Jones (2015; see also Woo et al., 2023). This
approach models, for each time-varying predictor, the time-invariant
country mean alongside the time-varying difference from this mean for
each country-year.

Finally, we addressed measurement uncertainty in the data for trust in
bureaucracy, income inequality, and the Corruption Perceptions Index by
incorporating it into the analysis (see Tai et al., 2024). The model was
estimated using the \texttt{brms} R package (Bürkner, 2017).

\begin{figure}
\centering
\pandocbounded{\includegraphics[keepaspectratio]{dcpo_trust_bureaucracy_files/figure-latex/resplot-1.pdf}}
\caption{Predicting Trust in Civil Servants Across Countries Over Time
\label{model}}
\end{figure}

The results are presented in Figure\nobreakspace{}\ref{model}. In terms
of economic outcomes, the increase in GDP per capita is associated with
a higher level of trust in civil servants in the short term. A
two-standard-deviation year-to-year change in per capita income
increases trust by 1 (95\% c.i.: 0 to 1.9) point. This significant but
relatively small effect suggests that GDP growth alone may not be
sufficient to sustain high levels of trust in civil servants, given that
economic growth may not transfer to effective resource management and
service delivery. Still, it indicates that the null findings of the
literature are the consequence of underpowered analyses: Houston et al.
(2016) examines only 21 country-year contexts and Morelock (2021) just
23, compared to the 1411 country-years that we are able to examine using
the new TCS dataset.

Unemployment exhibits a strong negative effect on trust in civil
servants, with a two-standard-deviation year-to-year increase in
unemployment decreasing trust in civil servants by 4 (95\% c.i.: -4.9 to
-3.2) points. As Houston et al. (2016) concludes---and contrary to null
findings elsewhere (see, e.g., Morelock, 2021)---high unemployment rates
can signal government inefficiency or failure to address critical
economic challenges, eroding trust in civil servants (see also Choi,
2018: 124).

In terms of physical insecurity, the mean number of intentional
homicides has a long-term negative impact on trust in civil servants. A
two-standard deviation increase in a country's mean number of homicides
is associated with 9.7 (95\% c.i.: -16.8 to -2.5) points less trust.
This result bolsters confidence that the finding of Uddin (2025), made
in the context of Bangladesh, is generalizable across countries: high
levels of violence and insecurity undermine public confidence in the
administration's competence to uphold law and order, diminishing trust
in its civil service.

This analysis provides no evidence that either inflation or income
inequality significantly affects trust in civil servants in the short or
long term when other factors are controlled, reinforcing conclusions of,
e.g., Houston et al. (2016) and Morelock (2021).

Regarding process, these results show that a higher government
effectiveness score has a strong positive long-term effect on trust in
civil servants. A two-standard deviation increase in a country's mean
effectiveness score is associated with 28.5 (95\% c.i.: 6.9 to 50.3)
points more trust across countries. This finding suggests that improving
the perceived quality of public services and policy formulation can lead
to a sustained increase in trust, replicating findings of Houston et al.
(2016) and Morelock (2021). Coproduction of public value provides a
compelling explanation, emphasizing how involving individuals in
policy-making processes can strengthen trust in the public sector
(Schmidthuber et al., 2021).

Although democratic capacity is found to mediate the relationship
between government openness and public trust (Schmidthuber et al.,
2021), we found the development of democracy is associated with less
trust in civil servants, both in the long run and in the short term.
This is contrary to the positive result found by Choi (2018). Critical
citizens in more democratic countries may trust civil servant only
critically and have higher expectations of them (Norris, 1999). However,
perceived corruption was not consistently associated with trust in civil
servants in either the long term or short run when controlling for the
other factors in our model; contrary to the smaller studies in this
literature, which concomitantly test single variables at a time, the
estimates presented here are close to zero.

This conceptual replication of the literature on the causes of trust in
civil servants sheds new light on some debates and reinforces some
conclusions. It finds evidence for arguments on both government outcomes
and government quality. However, government quality, measured by mean
government effectiveness---exerts larger effects over long term than
economic and public security outcomes like GDP per capita, unemployment,
and homicide rates. To sustain trust in civil servants, policymakers and
practitioners should prioritize institutional reforms that enhance
effectiveness and inclusiveness.

\section*{Discussion}\label{discussion}

The study of trust's role in policy processes has long been hindered by
the absence of comparable measures of trust in civil servants---key
actors in policy implementation---across countries and over time. This
gap has impeded efforts to identify the causes and consequences of
bureaucratic trust, resulting in mixed findings regarding its origins
and influence on policy support and implementation.

Using a state-of-the-art latent-variable model (Solt, 2020c), we develop
a dynamic, comparative measure of trust in civil servants that uncovers
significant variations both within and across countries. Our analysis
reveals that while economic performance and public security influence
trust in the short term, government quality and effectiveness in service
delivery and policy implementation have more enduring effects.

While this study focuses on the sources of trust in civil servants, the
publicly accessible TCS dataset also offers new opportunities to examine
critical policy questions. Researchers can investigate how varying
levels of trust affect policy implementation, citizen compliance with
regulations, and public acceptance of policy interventions. The
dataset's longitudinal nature enables analysis of how changes in trust
relate to policy reforms, implementation strategies, and policy
outcomes. These applications are particularly relevant for complex
policy challenges that require sustained public cooperation and support.

TCS is a useful resource for studying trust in civil servants, but it
also has limitations. As a country--year aggregate, it can mask
case-specific dynamics that are better interpreted with country
expertise. In addition, the model-based smoothing needed to address
sparse coverage implies measurement uncertainty; treating point
estimates as error-free can distort downstream inferences (Tai et al.,
2024). To support appropriate use, we release full posterior draws so
researchers can incorporate uncertainty directly in subsequent analyses
using established approaches (e.g., Caughey and Warshaw, 2018; Tai et
al., 2024; Woo et al., 2025).

Despite these limitations, TCS advances research on trust and governance
by enabling more comprehensive comparative analyses of the levels,
trends, and correlates of trust in civil servants across countries and
over time than has been feasible with prior data.

TCS offers a valuable resource for studying trust in civil servants, but
it has limitations. As an aggregate-level measure, it may obscure
subnational variation, polarization in trust, or case-specific dynamics
that require country expertise and qualitative or mixed methods to
illuminate. Second, the smoothing approach introduces measurement
uncertainty, and ignoring uncertainty risks distorting inferences (Tai
et al., 2024). We therefore release the full set of posterior draws,
allowing researchers to propagate measurement uncertainty in downstream
analyses using established strategies (see, e.g., Caughey and Warshaw,
2018; Tai et al., 2024; Woo et al., 2025).

Despite these limitations, TCS advances research on trust and governance
by enabling more comprehensive comparative analyses of the levels,
trends, and correlates of trust in civil servants across countries and
over time than has been feasible with prior data.

\section*{References}\label{references}
\addcontentsline{toc}{section}{References}

\phantomsection\label{refs}
\begin{CSLReferences}{1}{1}
\bibitem[\citeproctext]{ref-Adcock2001}
Adcock R and Collier D (2001) Measurement validity: A shared standard
for qualitative and quantitative research. \emph{American Political
Science Review} 95(3): 529--546.

\bibitem[\citeproctext]{ref-Anderson2003}
Anderson CJ and Tverdova YV (2003) Corruption, political allegiances,
and attitudes toward government in contemporary democracies.
\emph{American journal of political science} 47(1). Wiley Online
Library: 91--109.

\bibitem[\citeproctext]{ref-Bargain2020}
Bargain O and Aminjonov U (2020) Trust and compliance to public health
policies in times of COVID-19. \emph{Journal of public economics} 192.
Elsevier: 104316.

\bibitem[\citeproctext]{ref-Bell2015}
Bell A and Jones K (2015) Explaining fixed effects: Random effects
modeling of time-series cross-sectional and panel data. \emph{Political
Science Research and Methods} 3(1): 133--153.

\bibitem[\citeproctext]{ref-Bergquist2022}
Bergquist M, Nilsson A, Harring N, et al. (2022) Meta-analyses of
fifteen determinants of public opinion about climate change taxes and
laws. \emph{Nature Climate Change} 12(3). Nature Publishing Group UK
London: 235--240.

\bibitem[\citeproctext]{ref-Blind2007}
Blind PK (2007) Building trust in government in the twenty-first
century: Review of literature and emerging issues. In: \emph{7th global
forum on reinventing government building trust in government}, 2007, pp.
26--29. June, Vienna, Austria.

\bibitem[\citeproctext]{ref-Bouckaert2012}
Bouckaert G (2012) Trust and public administration.
\emph{Administration} 60(1): 91--115.

\bibitem[\citeproctext]{ref-Burkner2017}
Bürkner P-C (2017) {brms}: An {R} package for bayesian multilevel models
using stan. \emph{Journal of Statistical Software} 80: 1--28.

\bibitem[\citeproctext]{ref-Caughey2018}
Caughey D and Warshaw C (2018) Policy preferences and policy change:
Dynamic responsiveness in the american states, 1936--2014.
\emph{American Political Science Review} 112(2). Cambridge University
Press: 249--266.

\bibitem[\citeproctext]{ref-Caughey2019}
Caughey D, O'Grady T and Warshaw C (2019) Policy ideology in european
mass publics, 1981--2016. \emph{American Political Science Review}
113(3): 674--693.

\bibitem[\citeproctext]{ref-Choi2018}
Choi S (2018) Bureaucratic characteristics and citizen trust in civil
service in OECD member nations. \emph{International Area Studies Review}
21(2): 114--133.

\bibitem[\citeproctext]{ref-Claassen2019}
Claassen C (2019) Estimating smooth country--year panels of public
opinion. \emph{Political Analysis} 27(1): 1--20.

\bibitem[\citeproctext]{ref-Cole2024}
Cole M, Sun J, Jiang W, et al. (2024) Governmental capabilities and
responsiveness: Global investigations into CO2 emissions and
decarbonization. The ultimate super wicked problem! \emph{International
Journal of Public Administration}. Taylor \& Francis: 1--19.

\bibitem[\citeproctext]{ref-Coppedge2023}
Coppedge M, Gerring J, Knutsen CH, et al. (2023) V-dem
{[}country-year/country-date{]} dataset v13. Varieties of Democracy
(V-Dem) Project.

\bibitem[\citeproctext]{ref-Davidovic2024}
Davidovic D (Forthcoming) Quality of government and public support for
taxation for climate change mitigation: Evidence from 135 european
regions. \emph{European Political Science Review} FirstView. Cambridge
University Press: 1--26.

\bibitem[\citeproctext]{ref-Devine2024b}
Devine D (2024) Political trust and redistribution preferences.
\emph{Journal of European Public Policy}. Taylor \& Francis: 1--24.

\bibitem[\citeproctext]{ref-Devine2021}
Devine D, Gaskell J, Jennings W, et al. (2021) Trust and the coronavirus
pandemic: What are the consequences of and for trust? An early review of
the literature. \emph{Political Studies Review} 19(2). SAGE Publications
Sage UK: London, England: 274--285.

\bibitem[\citeproctext]{ref-Easton1965}
Easton D (1965) A systems analysis of political life. Epub ahead of
print 1965.

\bibitem[\citeproctext]{ref-Easton1975}
Easton D (1975) A re-assessment of the concept of political support.
\emph{British Journal of Political Science} 5(4): 435--457.

\bibitem[\citeproctext]{ref-Goodsell2006}
Goodsell CT (2006) A new vision for public administration. \emph{Public
Administration Review} 66(4). Wiley Online Library: 623--635.

\bibitem[\citeproctext]{ref-Harring2018}
Harring N (2018) Trust and state intervention: Results from a swedish
survey on environmental policy support. \emph{Environmental Science \&
Policy} 82. Elsevier: 1--8.

\bibitem[\citeproctext]{ref-Hooghe2011}
Hooghe M (2011) Why {There} is {Basically} {Only} {One} {Form} of
{Political} {Trust}. \emph{The British Journal of Politics \&
International Relations} 13(2): 269--275.

\bibitem[\citeproctext]{ref-Houston2016}
Houston DJ, Aitalieva NR, Morelock AL, et al. (2016) Citizen trust in
civil servants: A cross-national examination. \emph{International
Journal of Public Administration} 39(14): 1203--1214.

\bibitem[\citeproctext]{ref-Hu2025b}
Hu Y and Solt F (2025) Macrointerest across countries. \emph{British
Journal of Political Science} 55(e71): 1--10.

\bibitem[\citeproctext]{ref-Hu2025}
Hu Y, Tai YC, Ko H, et al. (2025) An incomplete recipe: One-dimensional
latent variables do not capture the full flavor of democratic support.
\emph{Research \& politics} 12(2). SAGE Publications Sage UK: London,
England: 20531680251341857.

\bibitem[\citeproctext]{ref-Kappler2024}
Kappler M, Verhoest K, Bach T, et al. (2024) What drives trust in
regulatory agencies? Probing the relevance of governmental level and
performance through a cross-national elite experiment on EU regulation.
\emph{European Political Science Review}. Cambridge University Press:
1--18.

\bibitem[\citeproctext]{ref-Kettl2000}
Kettl DF (2000) \emph{The Global Public Management Revolution: A Report
on the Transformation of Governance}. Brookings Institution Press.

\bibitem[\citeproctext]{ref-Kim2005}
Kim S-E (2005) The role of trust in the modern administrative state: An
integrative model. \emph{Administration \& Society} 37(5). Sage
Publications Sage CA: Thousand Oaks, CA: 611--635.

\bibitem[\citeproctext]{ref-Kolczynska2020}
Kolczynska M, Bürkner P-C, Kennedy L, et al. (2020) Trust in state
institutions in europe, 1989-2019. SocArXiv.
\texttt{https://osf.io/preprints/socarxiv/3v5g7/}.

\bibitem[\citeproctext]{ref-Macdonald2021}
Macdonald D (2021) Political trust and support for immigration in the
american mass public. \emph{British Journal of Political Science} 51(4).
Cambridge University Press: 1402--1420.

\bibitem[\citeproctext]{ref-Marien2011}
Marien S and Hooghe M (2011) Does political trust matter? {An} empirical
investigation into the relation between political trust and support for
law compliance. \emph{European Journal of Political Research} 50(2):
267--291.

\bibitem[\citeproctext]{ref-McGann2019}
McGann A, Dellepiane-Avellaneda S and Bartle J (2019) Parallel lines?
Policy mood in a plurinational democracy. \emph{Electoral Studies} 58:
48--57.

\bibitem[\citeproctext]{ref-Morelock2021}
Morelock AL (2021) In bureaucrats we trust? Good governance and trust in
civil servants. \emph{Public Administration Quarterly} 45(3). Southern
Public Administration Education Foundation: 315--337.

\bibitem[\citeproctext]{ref-Norris1999}
Norris P (1999) Introduction: The growth of critical citizens?
\emph{Critical citizens: Global support for democratic government}.
Oxford University Press Oxford: 1--27.

\bibitem[\citeproctext]{ref-Norris2002}
Norris P et al. (2002) \emph{Democratic Phoenix: Reinventing Political
Activism}. Cambridge University Press.

\bibitem[\citeproctext]{ref-Norris2011}
Norris P (2011) \emph{Democratic Deficit: Critical Citizens Revisited}.
Cambridge University Press.

\bibitem[\citeproctext]{ref-OECD2019}
OECD and Asian Development Bank (2019) \emph{Government at a Glance
Southeast Asia 2019}. Paris: OECD Publishing.

\bibitem[\citeproctext]{ref-OECD2025}
OECD and Asian Development Bank (2025) \emph{Government at a Glance:
Southeast Asia 2025}. Paris: OECD Publishing.

\bibitem[\citeproctext]{ref-Pemstein2023}
Pemstein D, Marquardt KL, Tzelgov E, et al. (2023) \emph{The v-dem
measurement model: Latent variable analysis for cross-national and
cross-temporal expert-coded data}. V-Dem Working Paper No. 21, 8th
edition. University of Gothenburg: Varieties of Democracy Institute.

\bibitem[\citeproctext]{ref-Rothstein2008}
Rothstein B and Stolle D (2008) The {State} and {Social} {Capital}: {An}
{Institutional} {Theory} of {Generalized} {Trust}. \emph{Comparative
Politics} 40(4): 441--459.

\bibitem[\citeproctext]{ref-Schmidthuber2021}
Schmidthuber L, Ingrams A and Hilgers D (2021) Government openness and
public trust: The mediating role of democratic capacity. \emph{Public
Administration Review} 81(1). Wiley Online Library: 91--109.

\bibitem[\citeproctext]{ref-Schmidthuber2023}
Schmidthuber L, Willems J and Krabina B (2023) Trust in public
performance information: The effect of data accessibility and data
source. \emph{Public Administration Review} 83(2). Wiley Online Library:
279--295.

\bibitem[\citeproctext]{ref-Shor2007}
Shor B, Bafumi J, Keele L, et al. (2007) A bayesian multilevel modeling
approach to time-series cross-sectional data. \emph{Political Analysis}
15(2): 165--181.

\bibitem[\citeproctext]{ref-Solt2020a}
Solt F (2020a) {DCPO}: Dynamic comparative public opinion. Available at
the Comprehensive R Archive Network (CRAN).
\texttt{https://CRAN.R-project.org/package=DCPO}.

\bibitem[\citeproctext]{ref-Solt2020}
Solt F (2020b) Measuring income inequality across countries and over
time: The standardized world income inequality database. \emph{Social
Science Quarterly} 101(3): 1183--1199.

\bibitem[\citeproctext]{ref-Solt2020c}
Solt F (2020c) Modeling dynamic comparative public opinion. SocArXiv.
\texttt{https://osf.io/\ preprints/socarxiv/d5n9p}.

\bibitem[\citeproctext]{ref-Solt2019}
Solt F, Hu Y and Tai Y (2019) {DCPOtools}: {Tools} for {Dynamic
Comparative Public Opinion}. Available at:
\url{https://github.com/fsolt/DCPOtools} (accessed 18 February 2022).

\bibitem[\citeproctext]{ref-Tai2022role}
Tai YC (2022) \emph{The role of trust in governance: Public health and
tax compliance in cross-national comparative perspective}. PhD thesis.
The University of Iowa.

\bibitem[\citeproctext]{ref-Tai2026}
Tai YC (2025) Introducing the trust in government (TrustGov) dataset: A
new resource for cross-national time-series trust research. SocArXiv.
Available at: \url{https://osf.io/preprints/socarxiv/k7caz_v1}.

\bibitem[\citeproctext]{ref-Tai2024}
Tai YC, Hu Y and Solt F (2024) Democracy, public support, and
measurement uncertainty. \emph{American Political Science Review}
118(1): 512--518.

\bibitem[\citeproctext]{ref-Uddin2025}
Uddin N (2025) Impact of crime and insecurity on citizen trust in public
institutions: Evidence from bangladesh. \emph{International Journal of
Public Administration} 48(4). Routledge: 238--249.

\bibitem[\citeproctext]{ref-Valgardhsson2021}
Valgarsson V, Stoker G, Devine D, et al. (2021) Disengagement and
political trust: Divergent pathways. \emph{Oxford handbook of political
participation}. Oxford University Press.

\bibitem[\citeproctext]{ref-VandeWalle2022}
Van de Walle S and Migchelbrink K (2022) Institutional quality,
corruption, and impartiality: The role of process and outcome for
citizen trust in public administration in 173 european regions.
\emph{Journal of Economic Policy Reform} 25(1): 9--27.

\bibitem[\citeproctext]{ref-VanRyzin2011}
Van Ryzin GG (2011) Outcomes, process, and trust of civil servants.
\emph{Journal of Public Administration Research and Theory} 21(4).
Oxford University Press: 745--760.

\bibitem[\citeproctext]{ref-Walker2017}
Walker RM, James O and Brewer GA (2017) Replication, experiments and
knowledge in public management research. \emph{Public Management Review}
19(9). Routledge: 1221--1234.

\bibitem[\citeproctext]{ref-Wilson2021}
Wilson MC and Knutsen CH (2022) Geographical coverage in political
science research. \emph{Perspectives on Politics} 20(3): 1024--1039.

\bibitem[\citeproctext]{ref-Woo2023a}
Woo B-D, Goldberg LA and Solt F (2023) Public gender egalitarianism: A
dataset of dynamic comparative public opinion toward egalitarian gender
roles in the public sphere. \emph{British Journal of Political Science}
53(2). Cambridge University Press: 766--775.

\bibitem[\citeproctext]{ref-Woo2025}
Woo B-D, Ko H, Tai YC, et al. (2025) Public support for gay rights
across countries and over time. \emph{Social Science Quarterly} 106(1).
Wiley Online Library: e13478.

\bibitem[\citeproctext]{ref-Wuttke2020}
Wuttke A, Schimpf C and Schoen H (2020) When the whole is greater than
the sum of its parts: On the conceptualization and measurement of
populist attitudes and other multidimensional constructs. \emph{American
Political Science Review} 114(2). Cambridge University Press: 356--374.

\bibitem[\citeproctext]{ref-yang2006performance}
Yang K and Holzer M (2006) The performance--trust link: Implications for
performance measurement. \emph{Public administration review} 66(1).
Wiley Online Library: 114--126.

\bibitem[\citeproctext]{ref-Yates1982}
Yates D (1982) \emph{Bureaucratic Democracy: The Search for Democracy
and Efficiency in American Government}. Harvard University Press.

\bibitem[\citeproctext]{ref-Zaki2022}
Zaki BL, Nicoli F, Wayenberg E, et al. (2022) In trust we trust: The
impact of trust in government on excess mortality during the COVID-19
pandemic. \emph{Public Policy and Administration} 37(2). SAGE
Publications Sage UK: London, England: 226--252.

\end{CSLReferences}

\pagebreak

\renewcommand{\baselinestretch}{1}
\selectfont
\maketitle
\selectfont

\pagenumbering{arabic}
\renewcommand*{\thepage}{A\arabic{page}}

\setcounter{figure}{0}
\renewcommand*{\thefigure}{A\arabic{figure}}

\vspace{-.5in}
\begin{center}
\begin{Large}
Appendices
\end{Large}
\end{center}

\section{Appendix A: Survey Items Used to Estimate Trust in Civil
Servants}\label{sec-app-survey}

National and cross-national surveys have often included questions
tapping trusting attitudes over the past half-century, but the resulting
data are both sparse, that is, unavailable for many countries and years,
and incomparable, generated by many different survey items. In all, we
identified 17 such survey items that were asked in no fewer than five
country-years in countries surveyed at least twice; these items were
drawn from 132 different survey datasets. These items are listed in the
table below, along with the dispersion (\(\alpha\)) and difficulty
(\(\beta\)) scores estimated for each from the DCPO model. Question text
may vary slightly across survey datasets, but not, roughly speaking, by
more than the translation differences across languages found within the
typical cross-national survey dataset. Lower values of dispersion
indicate questions that better identify publics with a higher level of
trust from those with lower. Items have one less difficulty score than
the number of response categories. Survey dataset codes correspond to
those used in the \texttt{DCPOtools} R package; they appear in
decreasing order of country-years contributed.

Together, the survey items in the source data were asked in 98 different
countries in at least two time points over 36 years, from 1973 to 2022,
yielding a total of 1,814 country-year-item observations. The TCS scores
of country-years with more observed items are likely to be estimated
more precisely. The estimates for country-years with fewer (or no)
observed items rely more heavily (or entirely) on the random-walk prior
and are therefore less certain.

\noindent Table A1: Indicators Used in the Unidimensional Latent
Variable Model of Democratic Support

\begingroup\fontsize{7}{9}\selectfont

\begin{longtable}[t]{>{\raggedright\arraybackslash}p{7em}>{\raggedright\arraybackslash}p{4em}>{\raggedright\arraybackslash}p{13em}>{\raggedright\arraybackslash}p{16em}>{\raggedright\arraybackslash}p{4em}>{\raggedright\arraybackslash}p{8em}>{\raggedright\arraybackslash}p{8em}}
\toprule
Survey
Item
Code & Country-Years & Question Text & Response Categories & Dispersion & Difficulties & Survey Dataset Codes\\
\midrule
\endfirsthead
\multicolumn{7}{@{}l}{\textit{(continued)}}\\
\toprule
Survey
Item
Code & Country-Years & Question Text & Response Categories & Dispersion & Difficulties & Survey Dataset Codes\\
\midrule
\endhead

\endfoot
\bottomrule
\endlastfoot
\cellcolor{gray!10}{trust4} & \cellcolor{gray!10}{614} & \cellcolor{gray!10}{And how much trust do you have in... Civil service / public administration} & \cellcolor{gray!10}{1 A great deal of trust / 2 Quite a lot of trust / 3 Not a lot of trust / 4 No trust at all} & \cellcolor{gray!10}{\num{0.88}} & \cellcolor{gray!10}{-1.39, 1.00, 3.86} & \cellcolor{gray!10}{evs, wvs, ases, lb, bsa, asianb, eass, itanes, kgss, sasianb, arabb}\\
trust2 & 348 & I would like to ask you a question about how much�trust�you have in certain�institutions. For each of the following�institutions, please tell me if you tend to�trustit or tend not to�trust�it? Civil service & 1 Tend to trust / 2 Tend not to trust & \num{1.13} & 0.79 & eb, cceb\\
\cellcolor{gray!10}{runswell4} & \cellcolor{gray!10}{197} & \cellcolor{gray!10}{How would you judge the current situation in each of the following? The way public administration runs in} & \cellcolor{gray!10}{1 Very good / 2 Rather good / 3 Rather bad / 4 Very bad} & \cellcolor{gray!10}{} & \cellcolor{gray!10}{} & \cellcolor{gray!10}{eb}\\
right5 & 109 & Most of the time we can trust people in government to do what is right. & 1 Strongly agree / 5 Strongly disagree & \num{0.85} & -1.11, 0.65, 2.00, 4.66 & issp, usgss\\
\cellcolor{gray!10}{best5} & \cellcolor{gray!10}{94} & \cellcolor{gray!10}{Most government administrators} & \cellcolor{gray!10}{1 Strongly agree / 5 Strongly disagree} & \cellcolor{gray!10}{\num{1.24}} & \cellcolor{gray!10}{-1.61, 0.53, 2.32, 5.65} & \cellcolor{gray!10}{issp, usgss, kgss}\\
image4 & 89 & Could you please tell me for each of the following, whether the term brings to mind something very positive, fairly positive, fairly negative or very negative.  Public administration & 1 Very positive / 2 Fairly positive / 3 Fairly negative / 4 Very negative & \num{0.83} & -1.80, 0.30, 3.03 & eb\\
\cellcolor{gray!10}{trustmun4} & \cellcolor{gray!10}{81} & \cellcolor{gray!10}{Generally speaking, the public administration of [CITY NAME] can be trusted} & \cellcolor{gray!10}{1 Strongly agree / 4 Strongly disagree} & \cellcolor{gray!10}{} & \cellcolor{gray!10}{} & \cellcolor{gray!10}{feb, lb}\\
trustff4 & 73 & Please look at this card and tell me how much confidence you have in each of the following groups, institutions or persons mentioned on the list: a lot, some, a little or no confidence? Firefighters & 1 A lot / 2 Some / 3 A little / 4 None &  &  & lb\\
\cellcolor{gray!10}{right4} & \cellcolor{gray!10}{60} & \cellcolor{gray!10}{You can generally trust the people who run our government to do what is right.} & \cellcolor{gray!10}{1 Strongly agree / 4 Strongly disagree} & \cellcolor{gray!10}{\num{0.54}} & \cellcolor{gray!10}{-0.58, 1.18, 3.44} & \cellcolor{gray!10}{asianb}\\
trust3 & 32 & Trust in Ministries and Government Agencies & 1 Very much / 2 Some / 3 Not very much & \num{1.02} & -0.02, 3.02 & usgss, jgss\\
\cellcolor{gray!10}{trusteuro2} & \cellcolor{gray!10}{30} & \cellcolor{gray!10}{If you would�trust�information�they provide on the changeover to the euro:�Public�administration} & \cellcolor{gray!10}{1 trust / 2 do not trust} & \cellcolor{gray!10}{} & \cellcolor{gray!10}{} & \cellcolor{gray!10}{feb}\\
trustpollution5 & 26 & How much�trust�do you have in each of the following groups to give you correct information about causes of pollution? Government departments & 1 A great deal of trust / 2 Quite a lot of trust / 3 Some trust / 4 Not much trust / 5 Hardly any trust &  &  & issp\\
\cellcolor{gray!10}{trust5} & \cellcolor{gray!10}{23} & \cellcolor{gray!10}{Confidence in the Civil Service?} & \cellcolor{gray!10}{1 Complete confidence / 2 A great deal of confidence / 3 Some confidence / 4 Very little confidence / 5 No confidence at all} & \cellcolor{gray!10}{\num{0.54}} & \cellcolor{gray!10}{-0.81, 0.41, 1.88, 3.38} & \cellcolor{gray!10}{issp, gles, fsdtrust, fsdeva, bsa}\\
trust11 & 16 & Now, thinking about institutions like Parliament, please use the scale of 0 to 10 to indicate how much trust you have for each of the following, where 0 is no trust and 10 is a great deal of trust: & 0 No trust / 10 A great deal of trust & \num{0.56} & -1.36, -1.01, -0.55, -0.08, 0.32, 1.04, 1.50, 2.15, 3.11, 3.96 & cid, fsdelection, bes\\
\cellcolor{gray!10}{right4a} & \cellcolor{gray!10}{10} & \cellcolor{gray!10}{In general, do you feel that the people in government are too often interested in looking after themselves, or do you feel that they can be trusted to do the right thing nearly all the time?} & \cellcolor{gray!10}{1 Usually look after themselves / 2 Sometimes look after themselves / 3 Sometimes can be trusted to do the right thing / 4 Usually can be trusted to do the right thing} & \cellcolor{gray!10}{\num{0.81}} & \cellcolor{gray!10}{0.48, 1.49, 2.93} & \cellcolor{gray!10}{aes}\\
interests7 & 8 & To what extent do you trust each of these political institutions to look after your interests? Civil servants & 1 No trust / 7 Great trust & \num{0.40} & -1.14, 0.20, 0.84, 1.53, 2.86, 3.46 & neb\\
\cellcolor{gray!10}{trustmun7} & \cellcolor{gray!10}{4} & \cellcolor{gray!10}{Please tell me for each institution or organisation how much trust you place in it.  The municipal administration} & \cellcolor{gray!10}{1 Absolutely no trust at all / 23456 / 7 A great deal of trust} & \cellcolor{gray!10}{} & \cellcolor{gray!10}{} & \cellcolor{gray!10}{allbus}\\*
\end{longtable}
\endgroup{}

\clearpage
\pagebreak

\begin{figure}
\centering
\pandocbounded{\includegraphics[keepaspectratio]{dcpo_trust_bureaucracy_files/figure-latex/obs_by_cy-1.pdf}}
\caption{Source Data Observations by Country and Year \label{obs_by_cy}}
\end{figure}

\clearpage
\pagebreak

\section{Appendix B: The DCPO Model}\label{sec-app-dcpo}

A number of recent studies have developed latent variable models of
public opinion based on cross-national survey data (Caughey et al.,
2019; see Claassen, 2019; Kolczynska et al., 2020; McGann et al., 2019).
To estimate trust in civil servants across countries and over time, we
employ the latest of these methods that is appropriate for data that is
not only incomparable but also sparse, the Dynamic Comparative Public
Opinion (DCPO) model elaborated in Solt (2020c).\footnote{ Solt (2020c)
  demonstrates that the DCPO model provides a better fit to survey data
  than the models put forward by Claassen (2019) or Caughey et al.
  (2019). The McGann et al. (2019) model depends on dense survey data
  unlike the sparse data on trust in civil servants described in the
  preceding section. Kolczynska et al. (2020) is the very most recent of
  these five works and builds on each of the others, but the MRP
  approach developed in that piece is suitable not only when the
  available survey data are dense but also when ancillary data on
  population characteristics are available, so it is similarly
  inappropriate to this application.} The DCPO model is a
population-level two-parameter ordinal logistic item response theory
(IRT) model with country-specific item-bias terms.

DCPO models the total number of survey responses expressing at least as
much trust in civil servants as response category \(r\) to each question
\(q\) in country \(k\) at time \(t\), \(y_{ktqr}\), out of the total
number of respondents surveyed, \(n_{ktqr}\), using the beta-binomial
distribution:

\begin{equation}
a_{ktqr} = \phi\eta_{ktqr} \label{eq:bb_a}
\end{equation} \begin{equation}
b_{ktqr} = \phi(1 - \eta_{ktqr}) \label{eq:bb_b}
\end{equation} \begin{equation}
y_{ktqr} \sim \textrm{BetaBinomial}(n_{ktqr}, a_{ktqr}, b_{ktqr}) \label{eq:betabinomial}
\end{equation}

where \(\phi\) represents an overall dispersion parameter to account for
additional sources of survey error beyond sampling error and
\(\eta_{ktqr}\) is the expected probability that a random person in
country \(k\) at time \(t\) answers question \(q\) with a response at
least as positive as response \(r\).\footnote{ The ordinal responses to
  question \(q\) are coded to range from 1 (expressing the least trust
  in civil servants) to \(R\) (expressing the most trust in civil
  servants), and \(r\) takes on all values greater than 1 and less than
  or equal to \(R\).}

This expected probability, \(\eta_{ktqr}\), is in turn estimated as
follows:

\begin{equation}
\eta_{ktqr} = \textrm{logit}^{-1}(\frac{\bar{\theta'}_{kt} - (\beta_{qr} + \delta_{kq})}{\sqrt{\alpha_{q}^2 + (1.7*\sigma_{kt})^2}}) \label{eq:dcpo}
\end{equation}

In this equation, \(\beta_{qr}\) represents the difficulty of response
\(r\) to question \(q\), that is, the degree of trust in civil servants
the response expresses. The \(\delta_{kq}\) term represents
country-specific item bias: the extent to which all responses to a
particular question \(q\) may be more (or less) difficult in a given
country \(k\) due to translation issues, cultural differences in
response styles, or other idiosyncrasies that render the same survey
item not equivalent across countries.\footnote{ Estimating
  \(\delta_{kq}\) requires repeated administrations of question \(q\) in
  country \(k\), so when responses to question \(q\) are observed in
  country \(k\) in only a single year, the DCPO model sets
  \(\delta_{kq}\) to zero by assumption, increasing the error of the
  model by any country-item bias that is present. Questions that are
  asked repeatedly over time in only a single country pose no risk of
  country-specific item bias, so \(\delta_{kq}\) in such cases are also
  set to zero.} The dispersion of question \(q\), its noisiness in
relation to our latent variable, is \(\alpha_{q}\). The mean and
standard deviation of the unbounded latent trait of trust in civil
servants are \(\bar{\theta'}_{kt}\) and \(\sigma_{kt}\), respectively.

Random-walk priors are used to account for the dynamics in
\(\bar{\theta'}_{kt}\) and \(\sigma_{kt}\), and weakly informative
priors are placed on the other parameters.\footnote{ The dispersion
  parameters \(\alpha_{q}\) are drawn from standard half-normal prior
  distributions, that is, the positive half of N(0, 1). The first
  difficulty parameters for each question, \(\beta_{q1}\), are drawn
  from standard normal prior distributions, and the differences between
  \(\beta\)s for each \(r\) for the same question \(q\) are drawn from
  standard half-normal prior distributions. The item-bias parameters
  \(\delta_{kq}\) receive normally-distributed hierarchical priors with
  mean 0 and standard deviations drawn from standard half-normal prior
  distributions. The initial value of the mean unbounded latent trait
  for each country, \(\bar{\theta'}_{k1}\), is assigned a standard
  normal prior, as are the transition variances
  \(\sigma_{\bar{\theta'}}^2\) and \(\sigma_{\sigma}^2\); the initial
  value of the standard deviation of the unbounded latent trait for each
  country, \(\sigma_{k1}\), is drawn from a standard lognormal prior
  distribution. The overall dispersion, \(\phi\), receives a somewhat
  more informative prior drawn from a gamma(4, 0.1) distribution that
  yields values that are well scaled for that parameter.} The dispersion
parameters \(\alpha_q\) are constrained to be positive and all survey
responses are coded with high values indicating more trust in civil
servants to fix direction. For each question \(q\) the difficulties for
increasing response categories \(r\) are constrained to be increasing.
The sum of \(\delta_{kq}\) across all countries \(k\) is set to zero for
each question \(q\):

\begin{equation}
\sum_{k = 1}^K \delta_{kq} = 0
\end{equation}

Finally, the logistic function is used to transform
\(\bar{\theta'}_{kt}\) to the unit interval and so give the bounded mean
of latent trust in civil servants, \(\bar{\theta}_{kt}\), which is our
parameter of interest here (see Solt, 2020c: 3--8).

The DCPO model accounts for the incomparability of different survey
questions with two parameters. First, it incorporates the
\emph{difficulty} of each question's responses, that is, how much trust
in civil servants is indicated by a given response. That each response
evinces more or less of our latent trait is most easily seen with regard
to the ordinal responses to the same question: strongly agreeing with
the statement ``Most government administrators (civil servants) can be
trusted to do what is best for the country,'' exhibits more trust in
civil servants than simply agreeing, which shows more trust than
responding ``disagree,'' which in turn is a more trusting response than
``strongly disagree.'' But this is also true across questions. Second,
the DCPO model accounts for each question's \emph{dispersion}, its
noisiness with regard to our latent trait. The lower a question's
dispersion, the better that changes in responses to the question map
onto changes in trust in civil servants. Together, the model's
difficulty and dispersion estimates work to generate comparable
estimates of the latent variable of trust in civil servants from the
available but incomparable source data.

To address the sparsity of the source data---the fact that there are
gaps in the time series of each country, and even many observed
country-years have only one or few observed items---DCPO uses simple
local-level dynamic linear models, i.e., random-walk priors, for each
country. That is, within each country, each year's value of trust in
civil servants is modeled as the previous year's estimate plus a random
shock. These dynamic models smooth the estimates of trust in civil
servants over time and allow estimation even in years for which little
or no survey data is available, albeit at the expense of greater
measurement uncertainty.

\end{document}
